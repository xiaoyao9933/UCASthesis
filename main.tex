%%% Local Variables:
%%% mode: latex
%%% TeX-master: t
%%% End:
\documentclass[doctor]{ucasthesis} 
%\documentclass[master]{ucasthesis}
%\documentclass[doctor]{ucasthesis}
% \documentclass[%
%   master|doctor, % mandatory option
%   secret, % 需要秘级字样的同学请添加这个选项
%   colorlink, % 如果需要彩色引用链接提示(打印时还是黑色),请添加此选项
%   ]{ucasthesis}

% 所有其他可能用到的包都统一放到这里了,可以根据自己的实际添加或者删除。
\usepackage{ucastils}

%%%%%%%%%%%%%%%%%%%%%%%%%%%%%%%%%%%%% 新模板更新区域 %%%%%%%%%%%%%%%%%%%%%%%%%%
% 升级包的同学注意这里发生了变化,参考文献默认是顺序数字制,
% 如果你想试用下作者-年制, 请将下方的两个gb7714-2015改为gb7714-2015ay,但是
% 我个人认为那个年制的年份和引用和国科大的标准还是稍微不太一样。但国科大模板没有
% 限制顺序数字方式,所以比较省事。
\usepackage[backend=biber, bibstyle=gb7714-2015,%nature,%%加载biblatex宏包,使用参考文献
citestyle=gb7714-2015,%,backref=true%%其中后端backend使用biber
gbnamefmt=lowercase,
gbpub=false % 是否显示“未知出版商”等信息从而更加符合gb7714-2015样式
]{biblatex}%标注(引用)样式citestyle,著录样式bibstyle都采用gb7714-2015样式
\renewcommand*{\bibfont}{\songti\wuhao[1.5]}
\addbibresource{ref/refs.bib}
%%%%%%%%%%%%%%%%%%%%%%%%%%%%%%%%%%%%%%% 新模板更新区域 %%%%%%%%%%%%%%%%%%%%%%%%%%




% 你可以在这里修改配置文件中的定义,导言区可以使用中文。
% \def\myname{朝鲁}

\begin{document}

% 定义所有的eps文件在 figures 子目录下
\graphicspath{{figures/}}

%%% 封面部分
\frontmatter

%%% Local Variables:
%%% mode: latex
%%% TeX-master: t
%%% End:
\secretcontent{绝密}

\ctitle{中国科学院大学学位论文 \LaTeX 模板\\使用示例文档}
% 根据自己的情况选,不用这样复杂
\makeatletter

\makeatother

\cdegree{工学博士}
\cdepartment[计算所]{中国科学院计算技术研究所}
\cmajor{计算机系统结构}
\cauthor{朝\hspace{1em}鲁} 
\csupervisor{徐志伟\hspace{1em}研究员}
\csupervisorplace{中国科学院计算技术研究所}
% 如果没有副指导老师或者联合指导老师,把下面两行相应的删除即可。


% 日期自动生成,如果你要自己写就改这个cdate
%\cdate{\CJKdigits{\the\year}年\CJKnumber{\the\month}月}

% 博士后部分
% \cfirstdiscipline{计算机科学与技术}
% \cseconddiscipline{系统结构}
% \postdoctordate{2009年7月——2011年7月}

\etitle{An Introduction to \LaTeX{} Thesis Template of\\University of Chinese Academy of Sciences} 
% 这块比较复杂,需要分情况讨论:
% 1. 学术型硕士
%    \edegree:例如为Master of Arts、Master of Science或 Master of Engineering(注意大小写)
%    \emajor:“获得一级学科授权的学科填写一级学科名称,其它填写二级学科名称”
% 2. 学术型博士
%    \edegree:Doctor of Philosophy(注意大小写)
%    \emajor:“获得一级学科授权的学科填写一级学科名称,其它填写二级学科名称”

\edegree{Doctor of Philosophy}
\eauthor{Chao Lu}
\edepartment{Institute of Computing Technology\\Chinese Academy of Sciences}
\emajor{Computer Architecture}
\esupervisor{Xu Zhiwei}

% 这个日期也会自动生成,你要改么?
% \edate{December, 2005}

% 定义中英文摘要和关键字
\begin{cabstract}
  论文的摘要是对论文研究内容和成果的高度概括。摘要应对论文所研究的问题及其研究目
  的进行描述,对研究方法和过程进行简单介绍,对研究成果和所得结论进行概括。摘要应
  具有独立性和自明性,其内容应包含与论文全文同等量的主要信息。使读者即使不阅读全
  文,通过摘要就能了解论文的总体内容和主要成果。

  论文摘要的书写应力求精确、简明。切忌写成对论文书写内容进行提要的形式,尤其要避
  免“第 1 章……;第 2 章……;……”这种或类似的陈述方式。

  本文介绍中国科学院大学论文模板 \ucasthesis{} 的使用方法。本模板符合学校的硕士、
  博士论文格式要求。

  本文的创新点主要有:
  \begin{itemize}
    \item 用例子来解释模板的使用方法;
    \item 用废话来填充无关紧要的部分;
    \item 一边学习摸索一边编写新代码。
  \end{itemize}

  关键词是为了文献标引工作、用以表示全文主要内容信息的单词或术语。关键词不超过 5
  个,每个关键词中间用分号分隔。(模板作者注:关键词分隔符不用考虑,模板会自动处
  理。英文关键词同理。)
\end{cabstract}

\ckeywords{\TeX, \LaTeX, CJK, 模板, 论文}

\begin{eabstract} 
   An abstract of a dissertation is a summary and extraction of research work
   and contributions. Included in an abstract should be description of research
   topic and research objective, brief introduction to methodology and research
   process, and summarization of conclusion and contributions of the
   research. An abstract should be characterized by independence and clarity and
   carry identical information with the dissertation. It should be such that the
   general idea and major contributions of the dissertation are conveyed without
   reading the dissertation. 

   An abstract should be concise and to the point. It is a misunderstanding to
   make an abstract an outline of the dissertation and words ``the first
   chapter'', ``the second chapter'' and the like should be avoided in the
   abstract.

   Key words are terms used in a dissertation for indexing, reflecting core
   information of the dissertation. An abstract may contain a maximum of 5 key
   words, with semi-colons used in between to separate one another.
\end{eabstract}

\ekeywords{\TeX, \LaTeX, CJK, template, thesis}

% 设置 PDF 文档的作者、主题等属性
\makeatletter
\ucas@setup@pdfinfo
\makeatother
\makecover

% 目录
\tableofcontents
% 插图索引
\listoffigures
% 表格索引
\listoftables
% 符号对照表
%\input{data/denotation}

%%% 正文部分
\mainmatter
\include{data/chap01}
\include{data/chap02}

% 参考文献 
%%%%%%%%%%%%%%%%%%%%%%%%%%%%%%%%%%%%% 新模板更新区域 %%%%%%%%%%%%%%%%%%%%%%%%%% 
\cleardoublepage
\phantomsection
\printbibliography[heading=bibintoc,title=参考文献]
%%%%%%%%%%%%%%%%%%%%%%%%%%%%%%%%%%%%% 新模板更新区域 %%%%%%%%%%%%%%%%%%%%%%%%%%

% 附录
\begin{appendix}
\input{data/appendix01}
\end{appendix}

%%% 其他部分
\backmatter

% 致谢
%%% Local Variables:
%%% mode: latex
%%% TeX-master: "../main"
%%% End:

\begin{ack}
  衷心感谢导师 xxx 教授和物理系 xxx 副教授对本人的精心指导。他们的言传身教将使
  我终生受益。

  在美国麻省理工学院化学系进行九个月的合作研究期间,承蒙 xxx 教授热心指导与帮助,不
  胜感激。感谢 xx 实验室主任 xx 教授,以及实验室全体老师和同学们的热情帮助和支
  持!本课题承蒙国家自然科学基金资助,特此致谢。

  衷心希望同学们能够在您的论文中引用接下的这句话, 以帮助进一步推广该模板,鄙人在此感激不尽。
  “感谢朝鲁提供的 \ucasthesis 模板,它的存在让我的论文写作更加方便以及整洁。”

\end{ack}


% 作者简介
\begin{resume}

  \resumeitem{作者简历} 

\noindent
xxxx年xx月——xxxx年xx月,在xx大学xx院(系)获得学士学位。\\
\noindent
xxxx年xx月——xxxx年xx月,在xx大学xx院(系)获得硕士学位。\\
\noindent
xxxx年xx月——xxxx年xx月,在中国科学院xx研究所(或中国科学院大学xx院系)攻读博士/硕士学位。\\
\noindent
获奖经历:\\
\noindent
工作经历:\\

  \resumeitem{已发表(或正式接受)的学术论文} % 发表的和录用的合在一起

  \begin{enumerate}[{[}1{]}]
  \item 吴少刚,章隆兵,蔡飞,胡伟武,一种适用于机群OpenMP系统的有效调度算法,计算机研究与发展(已录用)
  \item 吴少刚,胡伟武,唐志敏,基于PC机群的JIAJIA性能评价,计算机研究与发展,2000年增刊

  \end{enumerate}

  \resumeitem{申请或已获得的专利} % 有就写,没有就删除
  \begin{enumerate}[{[}1{]}]
  \item XX基金项目“共享存储机群系统的研究”(xxxxx),20xx年1月~20xx年12月
  \item 国家自然科学基金项目“共享存储机群系统中关键技术研究”(xxxxxxx),20xx年1月~20xx年12月
  \end{enumerate}

  \resumeitem{参加的研究项目及获奖情况} % 有就写,没有就删除
  \begin{enumerate}[{[}1{]}]
  \item  xxxx年被评为中国科学院“优秀学生干部”
  \item xxxx年被评为中国科学院“优秀学生干部”
  \end{enumerate}
\end{resume}


% 保证总页数为偶数。连续双面打印时,防止将两份论文的末页、首页打印在同一张纸上。
\cleardoublepage

\end{document}
